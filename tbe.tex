% LaTeX sources for introducing tfcb_log.
%
% dl@nosrv.org
\documentstyle[11pt]{article}
% Default margins are too wide all the way around.  I reset them here
\setlength{\topmargin}{-.5in}
\setlength{\textheight}{9in}
\setlength{\oddsidemargin}{.125in}
\setlength{\textwidth}{6.25in}
\begin{document}
\title{tfcb\_log By Example}
\author{$\Lambda$autcsh \\
\^\-C\^\-C\^\-C\^\-XquitqQq!dammit[esc]qwertyuiopasdfghjkl;:xwhat}
\renewcommand{\today}{March 3, 2012}
\maketitle
This paper demonstrates the installation and initialization process for tfcb\_log (v0.5.4). In addition to that, two stupid little use-cases were shown at the end of this document. 
\section {vi conf}
Edit the following variables within \texttt{/bin/make} to your own needs.
{\quotation \begin{tabular}{ll}
\texttt{WATCH\_DELAY\_SEC} (30)&- regulates the sleep timer, before the inbox is grabbed for input\\ 
&(watched mode).\\
\texttt{WATCH\_ONCE} (0)&- if set to ``1'', watching will stop after the first data was found.\\ 
&(watched mode).\\
\texttt{OUTPUT\_TRANSFER} (0)&- if set to ``1'', the transfer gets autom. triggered after the build process.\\
\texttt{OUTPUT\_LOGGING} (0)&- if set to ``1'', all the crap gets logged.\\
&(not implemented, yet!)\\
\texttt{BLOG\_HOST}&- defines your scp node.\\
\texttt{BLOG\_PORT}&- defines the listening port of your scp node.\\
\texttt{BLOG\_USER}&- your remote login username.\\
\texttt{BLOG\_PATH}&- remote path to your desired location.
\end{tabular}}
\newline\\\\
Edit the following variables within \texttt{/src/global.h} to your own needs.
{\quotation \begin{tabular}{ll}
\texttt{TMPL\_NAME} (main)&- set to your favourite template.\\
&(``main'', ``minimalistic'')\\
\end{tabular}}
\newline\\
\section {Compilation}
To start the compilation, follow these steps:\\\\
\texttt{cd src\\ make install clean\\\\}
The code will be compiled and the binary moves to \texttt{/}.
\\
In order to install the man page for tfcb\_log, please cp the ``tfcb\_log.1'' file within the src directory to into ``/usr/local/share/man/man1/'' by yourself.
\newpage
\texttt{make init\\\\}
This triggers the \texttt{/bin/make} script in order to initialize the required folder-structure. 
Here is the complete list of folder that {\em tfcb\_log} is going to create:\\
{\quotation \begin{tabular}{ll}
{\em arch}&- backup folder.\\
{\em inbox}&- the postman will grab your plainfiles here.\\
{\em posts}&- main node for the blog content.\\
{\em out}&- output folder for HTML \& CSS.\\
{\em latest}&- symlink to the newest folder under ``{\em posts/*}''.\\
{\em log}&- logfiles (not yet implemented).
\end{tabular}}
\bigskip\bigskip\newline
Afterwards the init script will ask you to activate the watched inbox. The nasty thing about that is, it will not start in the background. So at least, I prefer to say ``\texttt{n}'' in this case and activate it via ``\texttt{screen -m -d ./make w}''. I know this is an ugly workaround a for real process/thread-mgmt. But things happen and so you got to get happy with this.
\newline
\section {Synopsis}
{\quotation \begin{tabular}{ll}
\texttt{i } (init)&- initialize the entire blog.\\
\texttt{w } (watch)&- activate watched inbox.\\
\texttt{b } (build)&- build content from the inbox.\\
\texttt{t } (transfer)&- transfer content to your desired scp node.\\
\texttt{a } (archive)&- archive ``{\em posts/*}'' folder.\\
&(default ``daily'')\\
\texttt{ad} (archive)&- archive ``{\em posts/*}'' folder.\\
&(``daily'')\\
\texttt{am} (archive)&- archive ``{\em posts/*}'' folder.\\
&(default ``monthly'')\\
\texttt{ai} (archive)&- archive ``{\em posts/*}'' folder.\\
&(default ``incremental'')\\
\texttt{h } (help)&- prints help screen to stdout.\\
\end{tabular}}
\newline
\section {Echo -e "foo"}
Let's manually build some output by example:\\
\newline\texttt{
demo:tfcb\_log dl\$ echo "Hello, World." >> inbox/foo.txt
\newline
demo:tfcb\_log dl\$ cd bin/; ./make b
\\ ...
\newpage
-------------------------------
\\ tfcb\_log: administration script
\\ -------------------------------
\\ Version: 0.5.4 (2012-03-02) 
\\
\\ ii: check for latest softlink.
\\ ok: found today's folder.
\\ ii: moving blog post.
\\ ok: suceeded.
\\ ii: build output.
\\
\\ ii:	 HTML PROCESSING
\\
\\ ii:	 build index.html
\\ ok:	 write header.
\\ ok:	 begin to append posts.
\\ ii:	 appending post:
\\ ok:	 done.
\\ ok:	 all posts appended.
\\ ii:	 write footer.
\\ ok:	 built index.html, shalalala.
\\
\\ ii:	 RSS PROCESSING
\\
\\ ii:	 build rss.xml
\\ ok:	 write header.
\\ ok:	 begin to append elements.
\\ ii:	 appending element.
\\ ok:	 done.
\\ ok:	 all elements appended.
\\ ii:	 write footer.
\\ ok:	 built rss.xml, shalalala.
\\
\\ ok: succeeded.
\\ ok: nothing left to do.
}
\newpage
\section {Hit me!}
{\em TODO}\\
\bigskip\bigskip\newline
---dley, March 2012
{\quotation\
{\em Because C is my scripting language}\\
\author FETTEMAMA.ORG (2012)}
% International Herald Tribune, 22 May 1974, p8
\vfill\eject
\end{document}

